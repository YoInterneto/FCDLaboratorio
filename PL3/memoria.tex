\documentclass [a4paper] {article}

\usepackage[utf8]{inputenc}
\usepackage[spanish]{babel}
\usepackage{Sweave}
\usepackage{hyperref}


\title{PECL3 - Fundamentos de la ciencia de datos}
\author{Mario Adán Herrero \and Alberto González Martínez \and Branimir Stefanov Yanev \and Diego Gutiérrez Marco}

\begin{document}
\maketitle
\Sconcordance{concordance:memoria.tex:memoria.Rnw:%
1 42 1 1 5 4 0 1 2 1 1 13 0 1 2 2 1 1 3 5 0 1 2 12 1 1 3 2 0 1 1 1 2 1 0 2 1 7 0 1 2 2 1 1 8 %
13 0 1 2 7 1 1 2 1 0 1 1 5 0 1 1 3 0 1 2 2 1 1 6 11 0 1 2 5 1 1 3 2 0 1 1 10 0 1 2 1 0 1 1 7 %
0 1 2 2 1 1 3 2 0 1 1 6 0 1 2 4 1 1 2 1 0 1 1 1 8 12 0 1 2 7 1 1 3 2 0 2 1 11 0 1 2 2 1 1 3 2 %
0 2 1 10 0 1 4 2 0 1 3 1 0 1 1 11 0 1 2 5 1 1 6 11 0 1 2 18 1 1 4 3 0 1 2 1 1 14 0 1 2 2 1 1 %
3 5 0 1 2 12 1 1 3 2 0 1 1 1 2 1 0 2 1 7 0 1 2 2 1 1 8 14 0 1 2 7 1 1 2 1 0 1 1 5 0 2 1 6 0 1 %
2 2 1 1 6 8 0 1 2 3 1 1 3 2 0 1 1 10 0 1 2 1 0 1 1 7 0 1 2 2 1 1 3 2 0 1 1 6 0 1 2 4 1 1 2 1 %
0 1 1 1 8 12 0 1 2 8 1 1 3 2 0 2 1 13 0 1 4 2 0 1 3 1 0 1 1 14 0 1 2 5 1 1 6 18 0 1 2 17 1 1 %
2 4 0 1 2 1 1 1 2 4 0 1 2 1 1 1 2 1 0 2 1 11 0 1 2 1 1 1 2 1 0 1 1 6 0 1 2 1 1 1 2 8 0 1 2 1 %
1 1 2 4 0 1 2 1 1 1 2 4 0 1 2 14 1}


\begin{abstract}
En el siguiente documento se presentan los resultados y solución de la PECL3 del laboratorio de Fundamentos de la Ciencia de Datos.
Se realizará un análisis de clasificación de los datos utilizando R siguiendo los ejercicios realizados previamente en teoría.
Para la resolución del ejercicio utilizaremos el algoritmo de construcción de árboles de decisión de Hunt y también realizaremos un análisis de regresión lineal. Los datos calculados se mostrarán en este pdf utilizando las herramientas Sweave y TinyTex/MikTex.
\end{abstract}

\newpage

\tableofcontents

\newpage

\section{Introducción}
La práctica consta de dos partes:

En la primera parte se realizará un análisis de clasificación de Datos con R. Para esto utilizaremos el algoritmo de construcción de árboles de decisión de Hunt. Además utilizaremos los datos de la muestra para realizar un análisis de regresión lineal.

En la segunda parte, el grupo desarrollará un enunciado, y su posterior solución, de un ejercicio que contenga modificaciones del ejercicio hecho en clase, en el que se realice un análisis de clasificación supervisada con R, así como un análisis de regresión lineal.


\section{Ejercicio 1- Análisis de clasificación de datos. Calificaciones - Planetas} 
Utilizaremos las librerías tree y rpart para realizar la clasificación de los datos de las calificaciones de los alumnos en distintas partes de la asignatura (teoria, lab, práctica).
Realizaremos el análisis con ambas librerías y contrastaremos los resultados obtenidos.

\subsection{Apartado 1 - Árboles de decisión. Algoritmo Hunt}

En el primer apartado, se pide que, a partir de la muestra de las calificaciones de distintos alumnos “calificacionesMuestra.txt”, se obtenga la función de clasificación, utilizando para ellos, la medida de impureza Gini. 

La clasificación se realizará a partir de los siguientes atributos, siendo calificación el atributo clasificador:
\begin{enumerate}
\item Teoría
\item Laboratorio
\item Práctica
\item Calificación
\end{enumerate}

A continuación, mostramos la tabla con los datos:

\begin{Schunk}
\begin{Sinput}
> # Primero leemos los datos.
> calificaciones <- read.table("./data/calificacionesMuestra.txt")
> muestra <- data.frame(calificaciones)
> #Mostramos la tabla con los datos.
> muestra
\end{Sinput}
\begin{Soutput}
        Teoria Lab Pract Calificacion
suceso1      A   A     B           Ap
suceso2      A   B     D           Ss
suceso3      D   D     C           Ss
suceso4      D   D     A           Ss
suceso5      B   C     B           Ss
suceso6      C   B     B           Ap
suceso7      B   B     A           Ap
suceso8      C   D     C           Ss
suceso9      B   A     C           Ss
\end{Soutput}
\end{Schunk}

Para poder realizar el análisis cargamos las librerías \textit{tree} y \textit{rpart}

\begin{Schunk}
\begin{Sinput}
> # Cargamos librerías necesarias.
> library('tree')
> library('rpart')
\end{Sinput}
\end{Schunk}

Para realizar la clasificación, podemos utilizar cualquiera de las librerías cargadas, rpart y tree. Estas dos librerías utilizan para realizar la clasificación la medida de impureza Gini, con lo cual, no hace falta realizar ningún cambio en ninguna de sus llamadas.

\subsubsection{Librería rpart}

\begin{Schunk}
\begin{Sinput}
> clasificacionRpart <- rpart(Calificacion~., data=muestra, method="class", minsplit=1)
> #Mostramos el resultado
> print(clasificacionRpart)
\end{Sinput}
\begin{Soutput}
n= 9 

node), split, n, loss, yval, (yprob)
      * denotes terminal node

1) root 9 3 Ss (0.3333333 0.6666667)  
  2) Lab=A,B 5 2 Ap (0.6000000 0.4000000)  
    4) Pract=A,B 3 0 Ap (1.0000000 0.0000000) *
    5) Pract=C,D 2 0 Ss (0.0000000 1.0000000) *
  3) Lab=C,D 4 0 Ss (0.0000000 1.0000000) *
\end{Soutput}
\end{Schunk}

Como podemos observar la clasificación utilizando la librería rpart se realiza de forma correcta.

\subsubsection{Librería tree}

\begin{Schunk}
\begin{Sinput}
> calificacionTree <- tree(Calificacion~., data=muestra, mincut=1, minsize=2)
> print(calificacionTree)
\end{Sinput}
\begin{Soutput}
node), split, n, deviance, yval
      * denotes terminal node

1) root 9 NA NA *
\end{Soutput}
\end{Schunk}

Vemos que utilizando la librería tree el resultado de la clasificación es el mismo, esto se debe a que, como hemos dicho anteriormente, las dos librería utilizan como medida de impureza Gini por defecto.

\subsubsection{Apartado 2 - Análisis de regresión lineal}

Para este apartado, se pide que se realice un análisis de regresión lineal a partir de la muestra de los planetas “planetasMuestra.txt”.

\begin{Schunk}
\begin{Sinput}
> # Leemos y mostramos los datos
> planetas <- read.table("./data/planetasMuestra.txt")
> planetas
\end{Sinput}
\begin{Soutput}
           R   D
Mercurio 2.4 5.4
Venus    6.1 5.2
Tierra   6.4 5.5
Marte    3.4 3.9
\end{Soutput}
\end{Schunk}

Una vez cargados los datos pasamos a realizar el análisis de regresión, para ello obtendremos los coeficientes asociados entre la densidad (D) y el radio (R) de cada planeta.

\begin{Schunk}
\begin{Sinput}
> # Calculamos la regresión
> regresion = lm(D~R, data=planetas)
> print(regresion)
\end{Sinput}
\begin{Soutput}
Call:
lm(formula = D ~ R, data = planetas)

Coefficients:
(Intercept)            R  
     4.3624       0.1394  
\end{Soutput}
\end{Schunk}

Analizamos los resultados obtenidos en la regresión. Podemos observar que la recta de regresión que obtenemos (0.1394) es bastante mala ya que no se acerca nada a 1. La conclusión que sacamos es que no existe una correlación clara entre estas dos propiedades de los planetas ya que, por ejemplo, podemos observar que hay planetas como Mercurio con un radio muy pequeño (2.4) y otros mucho más grandes como Venus (R = 6.1), y sin embargo ambos tienen una densidad muy parecida (5.4 y 5.2, respectivamente).


\section{Ejercicio 2 - Análisis de clasificación de datos. Vehículos - Notas} 

Para este segundo ejercicio se deben realizar los mismo cálculos que para el ejercicio 1, pero con otras muestras y cambiando la forma en la que se calculan los resultados.
Las muestras utilizadas para este ejercicio serán “muestra1.txt” y “muestra2.txt”.

\subsection{Apartado 1 - Árboles de decisión. Algoritmo Hunt}

Para el primer apartado, a partir de una muestra con datos sobre las características de los vehículos, se pide realizar un análisis de clasificación de los datos sobre el atributo TipoVehiculo.

La clasificación se realizará a partir de los siguientes atributos, siendo calificación el atributo clasificador tipo vehículos:
\begin{enumerate}
\item Tipo carnet
\item Número de ruedas
\item Número de pasajeros
\item Tipo vehículo
\end{enumerate}

\begin{Schunk}
\begin{Sinput}
> # Leemos los datos de la primera muestra
> datos <- read.table("./data/muestra1.txt")
> muestra1 <- data.frame(datos)
> # Mostramos la tabla
> muestra1
\end{Sinput}
\begin{Soutput}
   TipoCarnet NumeroRuedas NumeroPasajeros TipoVehiculo
1           B            4               5        Coche
2           A            2               2         Moto
3           N            2               1    Bicicleta
4           B            6               4       Camion
5           B            4               6        Coche
6           B            4               4        Coche
7           N            2               2    Bicicleta
8           B            2               1         Moto
9           B            6               2       Camion
10          N            2               1    Bicicleta
\end{Soutput}
\end{Schunk}

\begin{Schunk}
\begin{Sinput}
> # Realizamos y mostramos la clasificación de los datos en función del campo TipoVehiculo
> clasificacion <- rpart(TipoVehiculo~., data=muestra1, method="class", minsplit=1)
> print(clasificacion)
\end{Sinput}
\begin{Soutput}
n= 10 

node), split, n, loss, yval, (yprob)
      * denotes terminal node

 1) root 10 7 Bicicleta (0.3000000 0.2000000 0.3000000 0.2000000)  
   2) TipoCarnet=N 3 0 Bicicleta (1.0000000 0.0000000 0.0000000 0.0000000) *
   3) TipoCarnet=A,B 7 4 Coche (0.0000000 0.2857143 0.4285714 0.2857143)  
     6) NumeroRuedas>=3 5 2 Coche (0.0000000 0.4000000 0.6000000 0.0000000)  
      12) NumeroRuedas>=5 2 0 Camion (0.0000000 1.0000000 0.0000000 0.0000000) *
      13) NumeroRuedas< 5 3 0 Coche (0.0000000 0.0000000 1.0000000 0.0000000) *
     7) NumeroRuedas< 3 2 0 Moto (0.0000000 0.0000000 0.0000000 1.0000000) *
\end{Soutput}
\end{Schunk}

Para poder realizar la visualización de los datos utilizamos la herramienta \textit{plot}

\begin{Schunk}
\begin{Sinput}
> plot(clasificacion, uniform=TRUE, main="Arbol de clasificacion vehiculos") 
> text(clasificacion, use.n=TRUE, all=TRUE, cex=.7, fancy=TRUE, fwidth=0.5, fheight=0.7)
\end{Sinput}
\end{Schunk}

\begin{figure}[!htbp]
  \centering
  \includegraphics[width=200px,keepaspectratio]{./arbolVehiculos.png}
  \caption{Árbol de decisión - Muestra 1}
  \label{fig:arbol}
\end{figure}

\newpage

La Figura 1. nos muestra el árbol de decisión construido. Analizándolo, podemos observar que la raíz será el atributo Tipo carnet, el cual ha sido elegido para ser la raíz utilizando la medida de la ganancia de información mediante la medida de impureza Gini. Después, el nodo hijo ha sido el atributo Número de ruedas y el nodo hijo de este, vuelve a ser Número de ruedas. Con estos nodos, ya es suficiente para clasificar todos los datos de la muestra.

\subsection{Apartado 2 - Análisis de regresión lineal}
Para este apartado, se pide que se realice un análisis de regresión lineal a partir “muestra2.txt”. El archivo de datos muestra2 ha sido generado mediante un script de python el cual genera 40 pares de valores (X, Y) entre 1 y 20, siendo X un número entero e Y un número decimal.

Para la resolución de este ejercicio dividiremos nuestra muestra en 4 muestras de 10 pares de valores cada una para estudiar la regresión.

\begin{Schunk}
\begin{Sinput}
> # Carga de los datos
> datos <- read.table("./data/muestra2.txt", header=TRUE)
> # Mostramos los datos 
> datos
\end{Sinput}
\begin{Soutput}
    X     Y
1   9  8.92
2   9  9.83
3  14 15.70
4   7  7.04
5  13 10.38
6  16 19.91
7  18 15.62
8  18 16.32
9  11 13.63
10 12 15.97
11 16 18.60
12  9 11.29
13  8 11.15
14  5  4.32
15  5  7.76
16  9  8.01
17  6  7.38
18 13 16.05
19 14 13.67
20  7  8.35
21 15 18.90
22  7 10.28
23  6  8.14
24 13 14.52
25 18 13.69
26  9  9.16
27  7 10.33
28 13 13.93
29 14 18.53
30 15 13.37
31 15 11.23
32 10 14.18
33  8  5.92
34  4  7.25
35  9 10.32
36 16 16.84
37 11  7.64
38  9  7.47
39 11 12.73
40  9 12.14
\end{Soutput}
\begin{Sinput}
> # Dividimos nuestra muestra en 4 muestras de menor tamaño
> muestras <- split(datos, factor(sort(rank(row.names(datos))%%4))) 
\end{Sinput}
\end{Schunk}

\subsubsection{Muestra 1}
Calcularemos la regresión para los datos de la primera muestra.

\begin{Schunk}
\begin{Sinput}
> # Datos de la primera muestra
> muestras[1]
\end{Sinput}
\begin{Soutput}
$`0`
    X     Y
1   9  8.92
2   9  9.83
3  14 15.70
4   7  7.04
5  13 10.38
6  16 19.91
7  18 15.62
8  18 16.32
9  11 13.63
10 12 15.97
\end{Soutput}
\end{Schunk}

Calculamos la regresión de la siguiente forma:

\begin{Schunk}
\begin{Sinput}
> # Calculamos la regresión de la muestra 1
> regresion1 = lm(Y~X, data=muestras[[1]])
> regresion1
\end{Sinput}
\begin{Soutput}
Call:
lm(formula = Y ~ X, data = muestras[[1]])

Coefficients:
(Intercept)            X  
     2.3091       0.8679  
\end{Soutput}
\end{Schunk}

\subsubsection{Muestra 2}
Calcularemos la regresión para los datos de la segunda muestra.

\begin{Schunk}
\begin{Sinput}
> # Datos de la segunda muestra
> muestras[2]
\end{Sinput}
\begin{Soutput}
$`1`
    X     Y
11 16 18.60
12  9 11.29
13  8 11.15
14  5  4.32
15  5  7.76
16  9  8.01
17  6  7.38
18 13 16.05
19 14 13.67
20  7  8.35
\end{Soutput}
\end{Schunk}

Calculamos la regresión de la siguiente forma:

\begin{Schunk}
\begin{Sinput}
> # Calculamos la regresión de la muestra 2
> regresion2 = lm(Y~X, data=muestras[[2]])
> regresion2
\end{Sinput}
\begin{Soutput}
Call:
lm(formula = Y ~ X, data = muestras[[2]])

Coefficients:
(Intercept)            X  
     0.9475       1.0555  
\end{Soutput}
\end{Schunk}

\subsubsection{Muestra 3}
Calcularemos la regresión para los datos de la tercera muestra.

\begin{Schunk}
\begin{Sinput}
> # Datos de la tercera muestra
> muestras[3]
\end{Sinput}
\begin{Soutput}
$`2`
    X     Y
21 15 18.90
22  7 10.28
23  6  8.14
24 13 14.52
25 18 13.69
26  9  9.16
27  7 10.33
28 13 13.93
29 14 18.53
30 15 13.37
\end{Soutput}
\end{Schunk}

Calculamos la regresión de la siguiente forma:

\begin{Schunk}
\begin{Sinput}
> # Calculamos la regresión de la muestra 3
> regresion3 = lm(Y~X, data=muestras[[3]])
> regresion3
\end{Sinput}
\begin{Soutput}
Call:
lm(formula = Y ~ X, data = muestras[[3]])

Coefficients:
(Intercept)            X  
     5.1626       0.6771  
\end{Soutput}
\end{Schunk}

\subsubsection{Muestra 4}
Calcularemos la regresión para los datos de la cuarta muestra.

\begin{Schunk}
\begin{Sinput}
> # Datos de la segunda muestra
> muestras[4]
\end{Sinput}
\begin{Soutput}
$`3`
    X     Y
31 15 11.23
32 10 14.18
33  8  5.92
34  4  7.25
35  9 10.32
36 16 16.84
37 11  7.64
38  9  7.47
39 11 12.73
40  9 12.14
\end{Soutput}
\end{Schunk}

Calculamos la regresión de la siguiente forma:

\begin{Schunk}
\begin{Sinput}
> # Calculamos la regresión de la muestra 4
> regresion4 = lm(Y~X, data=muestras[[4]])
> regresion4
\end{Sinput}
\begin{Soutput}
Call:
lm(formula = Y ~ X, data = muestras[[4]])

Coefficients:
(Intercept)            X  
     3.6999       0.6737  
\end{Soutput}
\end{Schunk}

Una vez calculada la regresión de las muestras, necesitamos una forma de poder visualizar el resultado, para ello, hemos optado por hacer una figura con las gráficas de regresión de cada una de las muestras.

\begin{Schunk}
\begin{Sinput}
> # Dibujamos las gráficas resultantes de las regresiones de cada muestra
> par(mfrow=c(2,2))
> # Muestra1
> plot(muestras[[1]]$X, muestras[[1]]$Y, main="Muestra 1", xlab="x", ylab="y")
> abline(regresion1, col="blue")
> # Muestra2
> plot(muestras[[2]]$X, muestras[[2]]$Y, main="Muestra 2", xlab="x", ylab="y")
> abline(regresion2, col="red")
> # Muestra3
> plot(muestras[[3]]$X, muestras[[3]]$Y, main="Muestra 3", xlab="x", ylab="y")
> abline(regresion3, col="orange")
> # Muestra4
> plot(muestras[[4]]$X, muestras[[4]]$Y, main="Muestra 4", xlab="x", ylab="y")
> abline(regresion4, col="black")
> mtext(expression(paste(bold("Datos de regresión de las muestras"))),
+          side = 3, line = -2, outer = TRUE)
\end{Sinput}
\end{Schunk}

Generamos los siguientes gráficos:
\begin{figure}[!htbp]
  \centering
  \includegraphics[width=200px,keepaspectratio]{./regresion.png}
  \caption{Cálculo de regresión de las muestras}
  \label{fig:regresiones}
\end{figure}

Analizando los diagramas de dispersión de las cuatro divisiones de la muestra inicial que teníamos, observamos que todas presentan una relación lineal positiva. También podemos decir que la recta de regresión para la muestra 2 es la más representativa y útil de todas las rectas de regresión que hemos obtenido, las cuales presentan una gran dispersión entre la recta y los puntos. Esto seguramente se deba a que los datos usados para este análisis se han generado de forma aleatoria.  

\clearpage


\section{Conclusiones}

Mediante esta práctica hemos empleado herramientas que permiten la realización de análisis de clasificación para un conjunto de datos mediante R. 
Se ha aprendido a utilizar el algoritmo de construcción de árboles de decisión de Hunt y a mostrar e interpretar este. También, se ha aprendido a realizar un análisis de regresión lineal e interpretar los resultados obtenidos.

\end{document}
