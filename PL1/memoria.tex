\documentclass [a4paper] {article}

\usepackage[utf8]{inputenc}
\usepackage[spanish]{babel}
\usepackage{Sweave}
\usepackage{hyperref}


\title{PECL1 - Fundamentos de la Ciencia de Datos}
\author{Mario Adán Herrero \and Alberto González Martínez \and Branimir Stefanov Yanev \and Diego Gutiérrez Marco}

\begin{document}
\maketitle
\Sconcordance{concordance:memoria.tex:memoria.Rnw:%
1 42 1 1 5 4 0 1 2 1 1 13 0 1 2 2 1 1 3 5 0 1 2 12 1 1 3 2 0 1 1 1 2 1 0 2 1 7 0 1 2 2 1 1 8 %
13 0 1 2 7 1 1 2 1 0 1 1 5 0 1 1 3 0 1 2 2 1 1 6 11 0 1 2 5 1 1 3 2 0 1 1 10 0 1 2 1 0 1 1 7 %
0 1 2 2 1 1 3 2 0 1 1 6 0 1 2 4 1 1 2 1 0 1 1 1 8 12 0 1 2 7 1 1 3 2 0 2 1 11 0 1 2 2 1 1 3 2 %
0 2 1 10 0 1 4 2 0 1 3 1 0 1 1 11 0 1 2 5 1 1 6 11 0 1 2 18 1 1 4 3 0 1 2 1 1 14 0 1 2 2 1 1 %
3 5 0 1 2 12 1 1 3 2 0 1 1 1 2 1 0 2 1 7 0 1 2 2 1 1 8 14 0 1 2 7 1 1 2 1 0 1 1 5 0 2 1 6 0 1 %
2 2 1 1 6 8 0 1 2 3 1 1 3 2 0 1 1 10 0 1 2 1 0 1 1 7 0 1 2 2 1 1 3 2 0 1 1 6 0 1 2 4 1 1 2 1 %
0 1 1 1 8 12 0 1 2 8 1 1 3 2 0 2 1 13 0 1 4 2 0 1 3 1 0 1 1 14 0 1 2 5 1 1 6 18 0 1 2 17 1 1 %
2 4 0 1 2 1 1 1 2 4 0 1 2 1 1 1 2 1 0 2 1 11 0 1 2 1 1 1 2 1 0 1 1 6 0 1 2 1 1 1 2 8 0 1 2 1 %
1 1 2 4 0 1 2 1 1 1 2 4 0 1 2 14 1}


\begin{abstract}
En el siguiente documento se presentan los resultados y solución de la PECL1 del laboratorio de Fundamentos de la Ciencia de Datos.
Se realizará un análisis estadísticos sobre diferentes datos proporcionados por el profesor y mediante funciones programadas en R se mostrarán los resultados en este pdf utilizando las herramientas Sweave y TinyTex/MikTex.

\end{abstract}

\newpage

\tableofcontents

\newpage

\section{Introducción} 
La práctica consta de dos ejercicios, en los cuales se utilizan archivos de datos de diferentes extensiones.\\
En este primer ejercicio se va a realizar un análisis, utilizando R, de diferentes archivos proporcionados por el profesor, estos archivos son “satelites.txt” y “cardata.sav”, los cuales contienen radios sobre los satélites de Urano (satelites.txt) y datos de automóviles cardata.sav).\\
En el segundo ejercicio, los datos del archivo “satelites.txt”, se han pasado a formato excel .xlsx, además se utilizarán nuevas librería y funciones para hallar las magnitudes.\\
Para estos dos ejercicios se hallarán las siguientes magnitudes:

\begin{enumerate}
\item Frecuencias.
\item Media aritmética.
\item Varianza.
\item Desviación típica.
\item Mediana.
\item Rango.
\item Cuartiles.
\item Cuantil 54.
\end {enumerate}

\section{Ejercicio 1.1 - Análisis de descripción de datos (satelites.txt)}
El primer paso a realizar es cargar los el archivo de texto que contiene los datos a estudiar y obtener el vector de datos que nos interesa, en este caso el vector radio. A continuación, se muestra el proceso:
\begin{Schunk}
\begin{Sinput}
> # Obtenemos y guardamos el archivo satelites.txt 
> #  de la carpeta donde está almacenado:
> satelites <- read.table("./data/satelites.txt")
> # Accedemos al valor de radio del archivo:
> radio <- satelites$Radio
> # Mostramos el valor del vector radio:
> radio
\end{Sinput}
\begin{Soutput}
 [1] 13 16 22 33 29 42 27 34 20 30 20 15
\end{Soutput}
\end{Schunk}

Una vez extraídos los valores de los radios y almacenados en la variable \textit{radio}, realizamos las operaciones que citamos anteriormente.

\subsection{Cálculo de frecuencias}
A continuación se muestran las fórmulas empleadas y los resultados de calcular las diferentes frecuencias sobre el vector radio obtenido anteriormente.

\subsubsection{Frecuencias absolutas}
Fórmula y resultado de las frecuencias absolutas:
\begin{Schunk}
\begin{Sinput}
> frecuencia_absoluta <- function(vector)(table(" "=vector))
> frecuencia_absoluta(radio)
\end{Sinput}
\begin{Soutput}
13 15 16 20 22 27 29 30 33 34 42 
 1  1  1  2  1  1  1  1  1  1  1 
\end{Soutput}
\end{Schunk}

\subsubsection{Frecuencias absolutas acumuladas}
Fórmula y resultado de las frecuencias absolutas acumuladas:
\begin{Schunk}
\begin{Sinput}
> frecuencia_absoluta_acumulada <- function(vector)(cumsum(frecuencia_absoluta(vector)))
> frecuencia_absoluta_acumulada(radio)
\end{Sinput}
\begin{Soutput}
13 15 16 20 22 27 29 30 33 34 42 
 1  2  3  5  6  7  8  9 10 11 12 
\end{Soutput}
\end{Schunk}

\subsubsection{Frecuencias relativas}
Fórmula y resultado de las frecuencias relativas:
\begin{Schunk}
\begin{Sinput}
> frecuencia_relativa <- function(vector)(round(frecuencia_absoluta(vector)/length(vector), 4))
> frecuencia_relativa(radio)
\end{Sinput}
\begin{Soutput}
    13     15     16     20     22     27     29     30     33     34     42 
0.0833 0.0833 0.0833 0.1667 0.0833 0.0833 0.0833 0.0833 0.0833 0.0833 0.0833 
\end{Soutput}
\end{Schunk}

\subsubsection{Frecuencias relativas acumuladas}
Fórmula y resultado de las frecuencias relativas acumuladas:
\begin{Schunk}
\begin{Sinput}
> frecuencia_relativa_acumulada <- function(vector)(round(frecuencia_absoluta_acumulada(vector)/length(vector), 4))
> frecuencia_relativa_acumulada(radio)
\end{Sinput}
\begin{Soutput}
    13     15     16     20     22     27     29     30     33     34     42 
0.0833 0.1667 0.2500 0.4167 0.5000 0.5833 0.6667 0.7500 0.8333 0.9167 1.0000 
\end{Soutput}
\end{Schunk}

\subsection{Media Aritmética}
Mediante la función que nos proporciona R calculamos la media aritmética redondeada a dos decimales. 
\begin{Schunk}
\begin{Sinput}
> media_aritmetica <- function(vector)(round(mean(vector), 2))
> media_aritmetica(radio)
\end{Sinput}
\begin{Soutput}
[1] 25.08
\end{Soutput}
\end{Schunk}

\subsection{Varianza}
Mediante la función que nos proporciona R calculamos la varianza redondeada a 2 decimales. 
\begin{Schunk}
\begin{Sinput}
> varianza <- function(vector)(round(var(vector), 2))
> varianza(radio)
\end{Sinput}
\begin{Soutput}
[1] 78.45
\end{Soutput}
\end{Schunk}

\subsection{Desviación típica}
Mediante la función que nos proporciona R calculamos la desviación típica redondeada a 2 decimales.
\begin{Schunk}
\begin{Sinput}
> desviacion_tipica <- function(vector)(round(sd(vector), 2))
> desviacion_tipica(radio)
\end{Sinput}
\begin{Soutput}
[1] 8.86
\end{Soutput}
\end{Schunk}


\subsection{Mediana}
Mediante la función que nos proporciona R calculamos la mediana. 
\begin{Schunk}
\begin{Sinput}
> mediana <- function(vector)(median(vector))
> mediana(radio)
\end{Sinput}
\begin{Soutput}
[1] 24.5
\end{Soutput}
\end{Schunk}

\subsection{Rango}
Creamos una función para realizar el cálculo del rango sobre el vector radio.
\begin{Schunk}
\begin{Sinput}
> rango <- function(vector)(max(vector)-min(vector))
> rango(radio)
\end{Sinput}
\begin{Soutput}
[1] 29
\end{Soutput}
\end{Schunk}

\subsection{Cuartiles y cuantil 54}
Mediante la función que nos proporciona R calculamos los cuartiles y el cuantil 54. 
\subsubsection{Cuartil 1}
Cálculo del cuartil 1:
\begin{Schunk}
\begin{Sinput}
> quantile(radio, 0.25)
\end{Sinput}
\begin{Soutput}
25% 
 19 
\end{Soutput}
\end{Schunk}

\subsubsection{Cuartil 2}
Cálculo del cuartil 2:
\begin{Schunk}
\begin{Sinput}
> quantile(radio, 0.50)
\end{Sinput}
\begin{Soutput}
 50% 
24.5 
\end{Soutput}
\end{Schunk}

\subsubsection{Cuartil 3}
Cálculo del cuartil 3:
\begin{Schunk}
\begin{Sinput}
> quantile(radio, 0.75)
\end{Sinput}
\begin{Soutput}
  75% 
30.75 
\end{Soutput}
\end{Schunk}

\subsubsection{Cuartil 4}
Cálculo del cuartil 4:
\begin{Schunk}
\begin{Sinput}
> quantile(radio, 1.00)
\end{Sinput}
\begin{Soutput}
100% 
  42 
\end{Soutput}
\end{Schunk}

\subsubsection{Cuantil 54}
Por otro lado, el Cuantil 54
\begin{Schunk}
\begin{Sinput}
> quantile(radio, 0.54)
\end{Sinput}
\begin{Soutput}
 54% 
26.7 
\end{Soutput}
\end{Schunk}

\section{Ejercicio 1.2 - Análisis de descripción de datos (cardata.sav)}

Fichero Cardata. Mismos valores, salvo Cuantil 54 y frecuencias. Para leer el archivo cardata, al ser .sav necesitamos el paquete "foreign" por lo que utilizamos la función library para cargarla ya que este paquete está instalado por defecto:
\begin{Schunk}
\begin{Sinput}
> # Cargamos las librerias:
> library(foreign)  # Libreria foreign
> library(xtable)
\end{Sinput}
\end{Schunk}

\begin{Schunk}
\begin{Sinput}
> # Obtenemos y guardamos el archivo cardata.sav 
> #  de la carpeta donde está almacenado:
> cardata <-read.spss("./data/cardata.sav")
> # Sacamos la variable con la que queremos trabajar de la tabla:
> mpg<-cardata$mpg
> # Como tenemos muchos valores nulos(NA), debemos reasignar mpg 
> #  para poder trabajar con los datos de forma correcta
> mpg<-mpg[!is.na(mpg)]
> #Mostramos los datos
> mpg
\end{Sinput}
\begin{Soutput}
  [1] 36.1 19.9 19.4 20.2 19.2 20.5 20.2 25.1 20.5 19.4 20.6 20.8 18.6 18.1 19.2
 [16] 17.7 18.1 17.5 30.0 30.9 23.2 23.8 21.5 19.8 22.3 20.2 20.6 17.0 17.6 16.5
 [31] 18.2 16.9 15.5 19.2 18.5 35.7 27.4 23.0 23.9 34.2 34.5 28.4 28.8 26.8 33.5
 [46] 32.1 28.0 26.4 24.3 19.1 27.9 23.6 27.2 26.6 25.8 23.5 30.0 39.0 34.7 34.4
 [61] 29.9 22.4 26.6 20.2 17.6 28.0 27.0 34.0 31.0 29.0 27.0 24.0 23.0 38.0 36.0
 [76] 25.0 38.0 26.0 22.0 36.0 27.0 27.0 32.0 28.0 31.0 43.1 20.3 17.0 21.6 16.2
 [91] 31.5 31.9 25.4 27.2 37.3 41.5 34.3 44.3 43.4 36.4 30.4 40.9 29.8 35.0 33.0
[106] 34.5 28.1 30.7 36.0 44.0 32.8 39.4 36.1 27.5 27.2 21.1 23.9 29.5 34.1 31.8
[121] 38.1 37.2 29.8 31.3 37.0 32.2 46.6 40.8 44.6 33.8 32.7 23.7 32.4 39.1 35.1
[136] 32.3 37.0 37.7 34.1 33.7 32.4 32.9 31.6 25.4 24.2 37.0 31.0 36.0 36.0 34.0
[151] 38.0 32.0 38.0 32.0
\end{Soutput}
\end{Schunk}

Tras esto calculamos las magnitudes calculadas en el los apartados anteriores, pero, en vez de usar el vector radio usamos mpg.

\subsection{Cálculo de frecuencias}
A continuación se muestran las fórmulas empleadas y los resultados de calcular las diferentes frecuencias sobre el vector radio obtenido anteriormente.

\subsubsection{Frecuencias absolutas}
Fórmula y resultado de las frecuencias absolutas:
\begin{Schunk}
\begin{Sinput}
> frecuencia_absoluta <- function(vector)(table(" "=vector))
> frecuencia_absoluta(mpg)
\end{Sinput}
\begin{Soutput}
15.5 16.2 16.5 16.9   17 17.5 17.6 17.7 18.1 18.2 18.5 18.6 19.1 19.2 19.4 19.8 
   1    1    1    1    2    1    2    1    2    1    1    1    1    3    2    1 
19.9 20.2 20.3 20.5 20.6 20.8 21.1 21.5 21.6   22 22.3 22.4   23 23.2 23.5 23.6 
   1    4    1    2    2    1    1    1    1    1    1    1    2    1    1    1 
23.7 23.8 23.9   24 24.2 24.3   25 25.1 25.4 25.8   26 26.4 26.6 26.8   27 27.2 
   1    1    2    1    1    1    1    1    2    1    1    1    2    1    4    3 
27.4 27.5 27.9   28 28.1 28.4 28.8   29 29.5 29.8 29.9   30 30.4 30.7 30.9   31 
   1    1    1    3    1    1    1    1    1    2    1    2    1    1    1    3 
31.3 31.5 31.6 31.8 31.9   32 32.1 32.2 32.3 32.4 32.7 32.8 32.9   33 33.5 33.7 
   1    1    1    1    1    3    1    1    1    2    1    1    1    1    1    1 
33.8   34 34.1 34.2 34.3 34.4 34.5 34.7   35 35.1 35.7   36 36.1 36.4   37 37.2 
   1    2    2    1    1    1    2    1    1    1    1    5    2    1    3    1 
37.3 37.7   38 38.1   39 39.1 39.4 40.8 40.9 41.5 43.1 43.4   44 44.3 44.6 46.6 
   1    1    4    1    1    1    1    1    1    1    1    1    1    1    1    1 
\end{Soutput}
\end{Schunk}

\subsubsection{Frecuencias absolutas acumuladas}
Fórmula y resultado de las frecuencias absolutas acumuladas:
\begin{Schunk}
\begin{Sinput}
> frecuencia_absoluta_acumulada <- function(vector)
+ (cumsum(frecuencia_absoluta(vector)))
> frecuencia_absoluta_acumulada(mpg)
\end{Sinput}
\begin{Soutput}
15.5 16.2 16.5 16.9   17 17.5 17.6 17.7 18.1 18.2 18.5 18.6 19.1 19.2 19.4 19.8 
   1    2    3    4    6    7    9   10   12   13   14   15   16   19   21   22 
19.9 20.2 20.3 20.5 20.6 20.8 21.1 21.5 21.6   22 22.3 22.4   23 23.2 23.5 23.6 
  23   27   28   30   32   33   34   35   36   37   38   39   41   42   43   44 
23.7 23.8 23.9   24 24.2 24.3   25 25.1 25.4 25.8   26 26.4 26.6 26.8   27 27.2 
  45   46   48   49   50   51   52   53   55   56   57   58   60   61   65   68 
27.4 27.5 27.9   28 28.1 28.4 28.8   29 29.5 29.8 29.9   30 30.4 30.7 30.9   31 
  69   70   71   74   75   76   77   78   79   81   82   84   85   86   87   90 
31.3 31.5 31.6 31.8 31.9   32 32.1 32.2 32.3 32.4 32.7 32.8 32.9   33 33.5 33.7 
  91   92   93   94   95   98   99  100  101  103  104  105  106  107  108  109 
33.8   34 34.1 34.2 34.3 34.4 34.5 34.7   35 35.1 35.7   36 36.1 36.4   37 37.2 
 110  112  114  115  116  117  119  120  121  122  123  128  130  131  134  135 
37.3 37.7   38 38.1   39 39.1 39.4 40.8 40.9 41.5 43.1 43.4   44 44.3 44.6 46.6 
 136  137  141  142  143  144  145  146  147  148  149  150  151  152  153  154 
\end{Soutput}
\end{Schunk}

\subsubsection{Frecuencias relativas}
Fórmula y resultado de las frecuencias relativas:
\begin{Schunk}
\begin{Sinput}
> frecuencia_relativa <- function(vector)
+ (round(frecuencia_absoluta(vector)/length(vector), 4))
> frecuencia_relativa(mpg)
\end{Sinput}
\begin{Soutput}
  15.5   16.2   16.5   16.9     17   17.5   17.6   17.7   18.1   18.2   18.5 
0.0065 0.0065 0.0065 0.0065 0.0130 0.0065 0.0130 0.0065 0.0130 0.0065 0.0065 
  18.6   19.1   19.2   19.4   19.8   19.9   20.2   20.3   20.5   20.6   20.8 
0.0065 0.0065 0.0195 0.0130 0.0065 0.0065 0.0260 0.0065 0.0130 0.0130 0.0065 
  21.1   21.5   21.6     22   22.3   22.4     23   23.2   23.5   23.6   23.7 
0.0065 0.0065 0.0065 0.0065 0.0065 0.0065 0.0130 0.0065 0.0065 0.0065 0.0065 
  23.8   23.9     24   24.2   24.3     25   25.1   25.4   25.8     26   26.4 
0.0065 0.0130 0.0065 0.0065 0.0065 0.0065 0.0065 0.0130 0.0065 0.0065 0.0065 
  26.6   26.8     27   27.2   27.4   27.5   27.9     28   28.1   28.4   28.8 
0.0130 0.0065 0.0260 0.0195 0.0065 0.0065 0.0065 0.0195 0.0065 0.0065 0.0065 
    29   29.5   29.8   29.9     30   30.4   30.7   30.9     31   31.3   31.5 
0.0065 0.0065 0.0130 0.0065 0.0130 0.0065 0.0065 0.0065 0.0195 0.0065 0.0065 
  31.6   31.8   31.9     32   32.1   32.2   32.3   32.4   32.7   32.8   32.9 
0.0065 0.0065 0.0065 0.0195 0.0065 0.0065 0.0065 0.0130 0.0065 0.0065 0.0065 
    33   33.5   33.7   33.8     34   34.1   34.2   34.3   34.4   34.5   34.7 
0.0065 0.0065 0.0065 0.0065 0.0130 0.0130 0.0065 0.0065 0.0065 0.0130 0.0065 
    35   35.1   35.7     36   36.1   36.4     37   37.2   37.3   37.7     38 
0.0065 0.0065 0.0065 0.0325 0.0130 0.0065 0.0195 0.0065 0.0065 0.0065 0.0260 
  38.1     39   39.1   39.4   40.8   40.9   41.5   43.1   43.4     44   44.3 
0.0065 0.0065 0.0065 0.0065 0.0065 0.0065 0.0065 0.0065 0.0065 0.0065 0.0065 
  44.6   46.6 
0.0065 0.0065 
\end{Soutput}
\end{Schunk}

\subsubsection{Frecuencias relativas acumuladas}
Fórmula y resultado de las frecuencias relativas acumuladas:
\begin{Schunk}
\begin{Sinput}
> frecuencia_relativa_acumulada <- function(vector)
+ (round(frecuencia_absoluta_acumulada(vector)/length(vector), 4))
> frecuencia_relativa_acumulada(mpg)
\end{Sinput}
\begin{Soutput}
  15.5   16.2   16.5   16.9     17   17.5   17.6   17.7   18.1   18.2   18.5 
0.0065 0.0130 0.0195 0.0260 0.0390 0.0455 0.0584 0.0649 0.0779 0.0844 0.0909 
  18.6   19.1   19.2   19.4   19.8   19.9   20.2   20.3   20.5   20.6   20.8 
0.0974 0.1039 0.1234 0.1364 0.1429 0.1494 0.1753 0.1818 0.1948 0.2078 0.2143 
  21.1   21.5   21.6     22   22.3   22.4     23   23.2   23.5   23.6   23.7 
0.2208 0.2273 0.2338 0.2403 0.2468 0.2532 0.2662 0.2727 0.2792 0.2857 0.2922 
  23.8   23.9     24   24.2   24.3     25   25.1   25.4   25.8     26   26.4 
0.2987 0.3117 0.3182 0.3247 0.3312 0.3377 0.3442 0.3571 0.3636 0.3701 0.3766 
  26.6   26.8     27   27.2   27.4   27.5   27.9     28   28.1   28.4   28.8 
0.3896 0.3961 0.4221 0.4416 0.4481 0.4545 0.4610 0.4805 0.4870 0.4935 0.5000 
    29   29.5   29.8   29.9     30   30.4   30.7   30.9     31   31.3   31.5 
0.5065 0.5130 0.5260 0.5325 0.5455 0.5519 0.5584 0.5649 0.5844 0.5909 0.5974 
  31.6   31.8   31.9     32   32.1   32.2   32.3   32.4   32.7   32.8   32.9 
0.6039 0.6104 0.6169 0.6364 0.6429 0.6494 0.6558 0.6688 0.6753 0.6818 0.6883 
    33   33.5   33.7   33.8     34   34.1   34.2   34.3   34.4   34.5   34.7 
0.6948 0.7013 0.7078 0.7143 0.7273 0.7403 0.7468 0.7532 0.7597 0.7727 0.7792 
    35   35.1   35.7     36   36.1   36.4     37   37.2   37.3   37.7     38 
0.7857 0.7922 0.7987 0.8312 0.8442 0.8506 0.8701 0.8766 0.8831 0.8896 0.9156 
  38.1     39   39.1   39.4   40.8   40.9   41.5   43.1   43.4     44   44.3 
0.9221 0.9286 0.9351 0.9416 0.9481 0.9545 0.9610 0.9675 0.9740 0.9805 0.9870 
  44.6   46.6 
0.9935 1.0000 
\end{Soutput}
\end{Schunk}

\subsection{Media Aritmética}
Mediante la función que nos proporciona R calculamos la media aritmética redondeada a dos decimales. 
\begin{Schunk}
\begin{Sinput}
> media_aritmetica <- function(vector)(round(mean(vector), 2))
> media_aritmetica(mpg)
\end{Sinput}
\begin{Soutput}
[1] 28.79
\end{Soutput}
\end{Schunk}

\subsection{Varianza}
Mediante la función que nos proporciona R calculamos la varianza redondeada a 2 decimales. 
\begin{Schunk}
\begin{Sinput}
> varianza <- function(vector)(round(var(vector), 2))
> varianza(mpg)
\end{Sinput}
\begin{Soutput}
[1] 54.42
\end{Soutput}
\end{Schunk}

\subsection{Desviación típica}
Mediante la función que nos proporciona R calculamos la desviación típica redondeada a 2 decimales.
\begin{Schunk}
\begin{Sinput}
> desviacion_tipica <- function(vector)(round(sd(vector), 2))
> desviacion_tipica(mpg)
\end{Sinput}
\begin{Soutput}
[1] 7.38
\end{Soutput}
\end{Schunk}

\subsection{Mediana}
Mediante la función que nos proporciona R calculamos la mediana. 
\begin{Schunk}
\begin{Sinput}
> mediana <- function(vector)(median(vector))
> mediana(mpg)
\end{Sinput}
\begin{Soutput}
[1] 28.9
\end{Soutput}
\end{Schunk}

\subsection{Rango}
Creamos una función para realizar el cálculo del rango sobre el vector radio.
\begin{Schunk}
\begin{Sinput}
> rango <- function(vector)(max(vector)-min(vector))
> rango(mpg)
\end{Sinput}
\begin{Soutput}
[1] 31.1
\end{Soutput}
\end{Schunk}

\subsection{Cuartiles y cuantil 54}
Mediante la función que nos proporciona R calculamos los cuartiles y el cuantil 54. 
\subsubsection{Cuartil 1}
Cálculo del cuartil 1:
\begin{Schunk}
\begin{Sinput}
> quantile(mpg, 0.25)
\end{Sinput}
\begin{Soutput}
  25% 
22.55 
\end{Soutput}
\end{Schunk}

\subsubsection{Cuartil 2}
Cálculo del cuartil 2:
\begin{Schunk}
\begin{Sinput}
> quantile(mpg, 0.50)
\end{Sinput}
\begin{Soutput}
 50% 
28.9 
\end{Soutput}
\end{Schunk}

\subsubsection{Cuartil 3}
Cálculo del cuartil 3:
\begin{Schunk}
\begin{Sinput}
> quantile(mpg, 0.75)
\end{Sinput}
\begin{Soutput}
   75% 
34.275 
\end{Soutput}
\end{Schunk}

\subsubsection{Cuartil 4}
Cálculo del cuartil 4:
\begin{Schunk}
\begin{Sinput}
> quantile(mpg, 1.00)
\end{Sinput}
\begin{Soutput}
100% 
46.6 
\end{Soutput}
\end{Schunk}

\subsubsection{Cuantil 54}
Por otro lado, el Cuantil 54
\begin{Schunk}
\begin{Sinput}
> quantile(mpg, 0.54)
\end{Sinput}
\begin{Soutput}
54% 
 30 
\end{Soutput}
\end{Schunk}

\section{Ejercicio 2 -  Análisis con R de fichero excel (satelites.xls} 
En el ejercicio 2 se van a calcular las mismas magnitudes que se hallaron en el ejercicio 1, pero en este caso, se utilizarán nuevas funciones y librerías. Además el archivo de donde se leen los datos será un .xlsx.\\
Para utilizar nuevas funciones se han instalado los siguientes paquetes; textit{xlsx} y \textit{pastecs}. Estos paquetes se han instalado de la siguiente forma:

\begin{verbatim}
install.packages("pastecs")
install.packages("xlsx")
\end{verbatim}

Tras esto, se ha procedido a leer los datos:

\begin{Schunk}
\begin{Sinput}
> # Cargamos la libreria:
> library(xlsx)
\end{Sinput}
\end{Schunk}

\begin{Schunk}
\begin{Sinput}
> # Obtenemos y guardamos el archivo "satelites.xlsx" de la carpeta donde está almacenado:
> satelites <- read.xlsx("./data/satelites.xlsx", 1)
> # Sacamos la variable con la que queremos trabajar de la tabla:
> radio <- satelites$radio
\end{Sinput}
\end{Schunk}

\subsection{Uso de data, frame, y table}


Una vez leídos, se ha utilizado \textit{table()} para calcular las frecuencias absolutas y \textit{prop.table()} para el cálculo de las frecuencias
relativas. Usamos \textit{data.frame()} para manipular las tablas usando nombres de columna y para mostrarlo en una sola tabla con nombres.

\subsection{Cálculo de frecuencias}
A continuación se muestran las fórmulas empleadas y los resultados de las diferentes frecuencias expresados en una tabla.

\subsubsection{Frecuencias absolutas}
Fórmula de las frecuencias absolutas:
\begin{Schunk}
\begin{Sinput}
> frecuencia_absoluta <- data.frame(table(radio))
\end{Sinput}
\end{Schunk}

\subsubsection{Frecuencias absolutas acumuladas}
Fórmula de las frecuencias absolutas acumuladas:
\begin{Schunk}
\begin{Sinput}
> frecuencia_absoluta_acumulada <- cumsum(frecAbs["Freq"])
\end{Sinput}
\end{Schunk}

\subsubsection{Frecuencias relativas}
Fórmula de las frecuencias relativas:
\begin{Schunk}
\begin{Sinput}
> frecuencia_relativa <- prop.table(frecAbs["Freq"])
\end{Sinput}
\end{Schunk}

\subsubsection{Frecuencias relativas acumuladas}
Fórmula de las frecuencias relativas acumuladas:
\begin{Schunk}
\begin{Sinput}
> frecuencia_relativa_acumulada <- cumsum(frecRel["Freq"])
\end{Sinput}
\end{Schunk}

A continuación, utilizamos data.frame() para mostrar todo lo calculado anteriormente, en un tabla, y poder presentarlo de forma distinta a los anteriores apartados:
\begin{Schunk}
\begin{Sinput}
> frecuencias <- data.frame(
+     frecuencia_absoluta["radio"],
+     frecuencia_absoluta["Freq"],
+     frecuencia_absoluta_acumulada["Freq"],
+     frecuencia_relativa ["Freq"],
+     frecuencia_relativa_acumulada ["Freq"]
+ )
> # Damos nombres a las entradas de la tabla
> names(frecuencias) <- c("Radio", "Absoluta", "Abs. acumulada",
+                         "Relativa", "Rel. acumulada")
> # Mostramos la tabla
> frecuencias
\end{Sinput}
\begin{Soutput}
   Radio Absoluta Abs. acumulada   Relativa Rel. acumulada
1     13        1              1 0.08333333     0.08333333
2     15        1              2 0.08333333     0.16666667
3     16        1              3 0.08333333     0.25000000
4     20        2              5 0.16666667     0.41666667
5     22        1              6 0.08333333     0.50000000
6     27        1              7 0.08333333     0.58333333
7     30        1              8 0.08333333     0.66666667
8     33        1              9 0.08333333     0.75000000
9     34        1             10 0.08333333     0.83333333
10    39        1             11 0.08333333     0.91666667
11    42        1             12 0.08333333     1.00000000
\end{Soutput}
\end{Schunk}

\subsection{Uso de pastecs}
Alternativamente, empleamos el paquete \textit{pastecs} para el cálculo de la media, mediana, mínimo, máximo y medidas de dispersión.

\subsubsection{Cálculo de media, mediana y medidas de dispersión } 
\begin{Schunk}
\begin{Sinput}
> # Cargamos la librería
> library(pastecs)
> # cargamos funciones sobre la tabla
> dispersion <- stat.desc(radio)[c("mean", "var", "std.dev", "range", "median")]
> # Establecemos nombres para la tabla 
> names(dispersion) <- c("Media", "Varianza", "Desv. típica", "Rango", "Mediana")
> # Mostramos por pantalla
> dispersion
\end{Sinput}
\begin{Soutput}
       Media     Varianza Desv. típica        Rango      Mediana 
   25.916667    93.901515     9.690279    29.000000    24.500000 
\end{Soutput}
\end{Schunk}

\subsubsection{Cuartiles y cuantil 54}
Utilizamos las funciones de pastecs para calcular y visulizar una tabla con información para los cuartiles y cuantil 54. 

\begin{Schunk}
\begin{Sinput}
> # Hallamos el cuantil 54
> cuantil54 <- quantile(radio, 0.54)
> # Añadimos a al lista de cuartiles el cuantil 54 y ordenamos
> cuartiles <- c(summary(radio), cuantil54)
> cuartiles <- cuartiles[order(unlist(cuartiles))]
> # Mostramos la tabla final
> print(cuartiles)
\end{Sinput}
\begin{Soutput}
    Min.  1st Qu.   Median     Mean      54%  3rd Qu.     Max. 
13.00000 19.00000 24.50000 25.91667 26.70000 33.25000 42.00000 
\end{Soutput}
\end{Schunk}


\section{Conclusiones}

Mediante esta práctica, nos hemos introducido al lenguaje de programación R y a algunas de sus herramientas, como es el caso de \textit{Latex}, sistema de preparación de documentos para composición de texto de alta calidad o de Sweave, componente de R que permite la integración de código en documentos escritos con \textit{Latex}.

Con el uso de todas estas herramientas, se ha podido realizar un análisis de un conjunto de datos, aplicando los conocimientos vistos en la teoría (calculo de distintas magnitudes como lo son las frecuancias, la media aritmética, o la varianza entre otros).

\end{document}
